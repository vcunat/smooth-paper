
%\begin{lem*}% \label{lem:smooth-bucket}

\begin{proof}

The smoothness (Definition~\ref{def:smooth}) over the domain $\interval a b$ gives us:
\begin{multline*}
	\exists\beta \; \forall c_1,c_2,c_3 \quad a \le c_1 < c_2 < c_3 \le b \quad \forall s \in \natural \\
		\Prob{  X \in \interval{ c_2 - \frac{c_3-c_1}{s^\alpha} }{c_2} \; \middle| \; X \in \interval{c_1}{c_3}  }
		\lespaces \frac{\beta s^{1-\delta}}{s}  \eqspaces  \beta s^{-\delta}.
\end{multline*}
We choose to cover the whole domain $\interval{c_1}{c_3} := \interval a b$. The conditioning can be removed because it is always fulfilled.
\[
	\exists\beta \; \forall c_2 \quad a < c_2 < b \quad \forall s \in \natural \quad
		\Prob{  X \in \interval{ c_2 - \frac{b-a}{s^\alpha} }{c_2}  }
		\lespaces \beta s^{-\delta}
\]

Now we consider splitting the domain into $k \ge n^{\alpha / \delta}$ equally long intervals.
Let us choose $c_2$ as the endpoint of an arbitrary%
	\footnote{There is a technical difficulty with the last interval, because Definition~\ref{def:smooth} (taken from referred papers) does not allow us to choose $c_2 = b$ for some unknown reason.
	However, we can choose $c_2 := b-1$, so only the maximal value is not covered. Adding a single key to the last interval separately then does not affect the validity of the implied Lemma%~\ref{lem:smooth-bucket}
	.
	}
interval $I$ and choose $s:= \lfloor n^{1 / \delta} \rfloor$, so
$ s^\alpha \le n^{\alpha / \delta} \le k $ and thus the above probability covers at least the whole interval $I$. That gives us:
\[	\Prob{X \in I} \lespaces \beta s^{-\delta}
	\eqspaces \beta {\lfloor n^{1 / \delta} \rfloor}^{-\delta}
	\lespaces \beta \left( n^{1 / \delta} - 1 \right) ^{-\delta}.
\] \[
\text{Since } \lim_{n \to \infty}
	\frac{ ( n^{1 / \delta} ) ^{-\delta} }{ ( n^{1 / \delta} - 1 ) ^{-\delta} }
	= 1 \text{, we conclude that } \Prob{X \in I} \text{ is } \OO(1/n).
\]
%
The input keys are chosen independently, so the number of keys in $I$ is given by the binomial distribution, and its expected value
$n \Prob{X \in I}$ is in $\OO(1)$.
\end{proof}

\begin{rem*}
We showed that the number of keys in an interval is expected to be constant, but the tail of binomial distribution can be bounded even more tightly, e.\,g.~by Chernoff bounds~\cite[chapter~4.1]{randomAlgs}, to guarantee that high values are exponentially rare.
We do not elaborate on a finer analysis, because we feel there is a more pressing problem that we do not solve in this paper~-- the unrealistic (unmotivated) character of the used input model.
We use it for our analysis to enable comparison with known structures, as we have only found one published structure that uses a different model~\cite{DemaineJP04} and moreover the bounds implied in that case seem relatively weak.
\end{rem*}

